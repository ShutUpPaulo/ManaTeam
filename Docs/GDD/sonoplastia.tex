\part*{Sonoplastia}

\section{Músicas e Efeitos Sonoros}

As músicas e os efeitos sonoros serão essenciais para a imersão no jogo, em conjunto com a arte. Elas possuem o objetivo de causar no jogador um sentimento de tensão constante, com melodias mais lentas e efeitos sonoros específicos de cada fase, que são vinculados à partes da história contida em cada fase. 

Nessa seção estão descritos todos os arquivos de áudio que serão utilizados no jogo. Cada fase terá uma música específica e um conjunto de efeitos sonoros padrões para todas as fases, além dos efeitos específicos, descritos no tópico \ref{efx}. 
\subsection{Músicas}
\begin{list}{}{}
\item\textbf{Nome:} abertura.ogg;

\textbf{Quando será utilizado:} música de abertura do jogo, será iniciada quando o jogo começar a ser executado até o jogador sair do menu principal; \\

\item\textbf{Nome:} fase1.ogg;

\textbf{Quando será utilizado:} será executada durante toda a fase 1; \\

\item\textbf{Nome:} fase2.ogg;

\textbf{Quando será utilizado:} será executada durante toda a fase 2; \\

\item\textbf{Nome:} fase3.ogg;

\textbf{Quando será utilizado:} será executada durante toda a fase 3; \\

\item\textbf{Nome:} fase4.ogg;

\textbf{Quando será utilizado:} será executada durante toda a fase 4; \\

\item\textbf{Nome:} fase5.ogg;

\textbf{Quando será utilizado:} será executada durante toda a fase 5; \\

\item\textbf{Nome:} ending.ogg;

\textbf{Quando será utilizado:} será executada ao final do jogo, quando o jogador terminar o jogo; \\


\end{list}
\subsection{\label{efx}Efeitos sonoros gerais}
\begin{list}{}{}
\item\textbf{Nome: } navegacaomenu.ogg;

\textbf{Descrição: } som curto de navegação nos menus;

\textbf{Quando será utilizado: } som de navegação no menu in game e no menu principal. Será acionado toda vez que o jogador selecionar um dos itens do menu; \\


\item\textbf{Nome:} key.ogg;

\textbf{Descrição:} som semelhante a um badalar de um sino;

\textbf{Quando será utilizado:} sempre que o jogador pegar a chave em alguma fase; \\

\item\textbf{Nome:} finalFase.ogg;

\textbf{Descrição:} toque curto de sanfona;

\textbf{Quando será utilizado:} som executado ao final da fase, quando o jogador inserir a chave na porta e passar para a próxima fase; \\

\item\textbf{Nome:} gritoGuarda.ogg;

\textbf{Descrição:} um grito em alemão com voz masculina;

\textbf{Quando será utilizado:} ao avistar o jogador, o guarda irá se espantar e gritar, chamando a atenção do jogador;  \\

\item\textbf{Nome:} danger.ogg;

\textbf{Descrição:} som curto demonstrando perigo;

\textbf{Quando será utilizado:} ao avistar o jogador, o guarda irá gritar para o jogador e esse som será executado junto ao grito, indicando o espanto do guarda e avisando o player do perigo iminente; \\

\item\textbf{Nome:} capturado.ogg;

\textbf{Descrição:} som de fade out sinalizando a derrota do personagem;

\textbf{Quando será utilizado:} Quando o player for capturado pelos guardas da fase; \\

\item\textbf{Nome:} gameover.ogg;

\textbf{Descrição:} som grave com fade out indicando que a partida acabou;

\textbf{Quando será utilizado:} Quando o player perder todas as vidas e a tela de game over aparecer; \\


\item\textbf{Nome:} correndo.ogg;

\textbf{Descrição:} som de uma pessoa correndo;

\textbf{Quando será utilizado:} toda vez que o personagem correr; \\

\item\textbf{Nome:} ataquesoco.ogg;

\textbf{Descrição:} som abafado representando um soco;

\textbf{Quando será utilizado:} toda vez que o personagem desferir um soco em alguém; \\

\item\textbf{Nome:} ataquefaca.ogg;

\textbf{Descrição:} som de algo como vísceras se rasgando;

\textbf{Quando será utilizado:} toda vez que o personagem desferir uma facada em alguém; \\

\item\textbf{Nome:} ataquegarrafa.ogg;

\textbf{Descrição:} som de vidro sendo quebrado;

\textbf{Quando será utilizado:} toda vez que o personagem desferir uma garrafada em alguém;  \\
\end{list}

\subsection{\label{efxF}Efeitos sonoros específicos de cada fase}
\begin{list}{}{}
\item\textbf{\textit{Fase 1}}
\item\textbf{Nome:} alarme.ogg;

\textbf{Descrição:} som de um alarme de presídio;

\textbf{Quando será utilizado:} No início da fase, durante 15 segundos \\

\item\textbf{Nome:} gritoKill.ogg;

\textbf{Descrição:} alemão dando grito de ordem;

\textbf{Quando será utilizado:} Em momentos aleatórios, juntamente com o efeito descrito a seguir; \\

\item\textbf{Nome:} rifleAtirando.ogg;

\textbf{Descrição:} som de metralhadora atirando;

\textbf{Quando será utilizado:} em momentos aleatórios, o jogador ouvirá um grito de ordem em alemão e logo em seguida o som de metralhadora; \\

\item\textbf{\textit{Fase 2}}
\item\textbf{Nome:} criancasRindo.ogg;

\textbf{Descrição:} som de crianças rindo;

\textbf{Quando será utilizado:} Em momentos aleatórios, o jogador escutará o som de crianças sorrindo, que o lembrará do episódio com o soldado criança; \\

\item\textbf{\textit{Fase 3}}
\item\textbf{Nome:} gas.ogg;

\textbf{Descrição:} som de gás escapando;

\textbf{Quando será utilizado:} em momentos aleatórios da fase, o personagem irá ver a sala em que se encontra se encher de gás e ouvirá o barulho do gás escapando;  \\

\item\textbf{Nome:} gritoGas.ogg;

\textbf{Descrição:} som de gritos de pessoas agonizando;

\textbf{Quando será utilizado:} em conjunto com o som do gás escapando, o personagem escutará gritos de agonia que lembrará o personagem das pessoas que morreram na câmara de gás; \\

\textbf{\textit{Fase 4}}
\item\textbf{Nome:} explosão.ogg;

\textbf{Descrição:} som de explosão;

\textbf{Quando será utilizado:} em momentos aleatórios da fase, o personagem escutará o som de uma explosão seguido do efeito descrito a seguir; \\

\item\textbf{Nome:} desmoronamento.ogg;

\textbf{Descrição:} som de um construto sendo desmoronado;

\textbf{Quando será utilizado:} em conjunto com o efeito sonoro descrito anteriormente; \\

\textbf{\textit{Fase 5}}
\item\textbf{Nome:} gritoEsposa.ogg;

\textbf{Descrição:} grito de agonia feminino;

\textbf{Quando será utilizado:} Em momentos aleatórios para referenciar a esposa de Edmond que morreu em uma explosão na Itália; \\

\end{list}