\part*{O mundo do jogo e seus personagens}
\section{A história do jogo}

   Início da 2a Guerra Mundial. A Alemanha invadiu a Polônia, levando a França e o Reino Unido a declarar guerra à Alemanha. O personagem, Edmond Gauthier - um sargento do exército francês - é um homem com um forte senso de honra que lidera bravamente seu exército frente ao inimigo alemão.
   
Em uma missão de resgate na Polônia, Edmond Gauthier e seu exército sofrem uma emboscada, planejada com o principal objetivo de impedi-los de resgatar os reféns judeus que seriam levados ao campo de concentração alemão ainda em construção. Embora a emboscada tenha causado algumas baixas, Edmond consegue alcançar os reféns, porém tarde demais. Ao se aproximar do local, ele percebe que os reféns já haviam sido executados e estavam sendo removidos do caminhão que continha a câmara de gás, uns dos últimos remanescentes da 1ª Guerra Mundial.
   
    Mesmo tendo falhado na missão, o exército francês ataca os alemães, numa espécie de revanche e  aprisiona-os. O superior de Edmond define então que os soldados nazistas aprisionados devem ser fuzilados, o que abala Edmond. Embora seja claro que os nazistas são um mal crescente, o código de honra de Edmond não o permite assassinar homens indefesos, sem a menor chance de revidarem ou sequer se protegerem. No entanto, como a ordem veio de um superior, Edmond é obrigado a obedecer e fuzila os nazistas aprisionados. Após essa missão mal sucedida, e com a crescente ameaça alemã, Edmond resolve voltar ao seu lar na França para se despedir de sua família e os mandarem para a Inglaterra, onde supostamente estariam mais protegidos.
    
    Em 1940, a Alemanha contorna a Linha Magnot - barreira de defesa que cobria toda a fronteira entre a França e a Alemanha, criada pela França após a 1a Guerra Mundial para evitar ataques surpresa e garantir mais tempo de resposta aos franceses em caso de um ataque frontal – vindo pelas densas florestas de Ardenas, único local desprotegido, pois acharam que os tanques alemães não conseguiriam atravessar a floresta. O General Weygand, recentemente nomeado, tomou ações imediatas para conter os alemães. Edmond então é enviado juntamente com o exército francês para conter os alemães. Durante a missão, Dante Chevalier, amigo de Edmond fica sob a mira do fogo inimigo. Edmond, logo parte em sua defesa, abatendo um soldado inimigo de pequeno porte. Ao investigar o corpo do inimigo, Edmond percebe que o soldado era uma criança, de aparentemente menos que 15 anos, o que o deixa perturbado. 
    
    Dois dias depois, ainda tentando conter os alemães, seu esconderijo é atacado por um tanque, desmoronando sobre suas cabeças. Edmond percebe então que Pierre ficou preso nos escombros. Ao tentar resgatar o amigo, acaba notando que o mesmo já se encontrava praticamente sem pulso e desacordado. Edmond decide então, mesmo contra sua vontade, abandonar o amigo e os demais soterrados e foi ao encalço dos alemães.
    
    O exército francês recua enquanto aguarda reforços e, ao receber informes da situação da Europa, descobre que a cidade inglesa onde sua família encontrava-se refugiada foi fortemente bombardeada pelos exércitos alemães. Esse era então o fim para Edmond Gauthier. Nada mais que ele amava lhe restava, além de amargas lembranças dos últimos e turbulentos meses. No entanto, uma missão mais precisava ser cumprida: parar os alemães.
    
    Os alemães, por outro lado, estavam muito fortalecidos e, mesmo com todo o empenho do exército francês e inglês, eles conseguiram avançar adentro do território francês. Edmond e mais alguns oficiais são então aprisionados e torturados em longos interrogatórios. Após obter as informações que necessitavam, os alemães aprisionaram os oficiais no campo de concentração de Bergen- Belsen (ou apenas Belsen) para utilizá-los depois como moeda de troca por oficiais alemães.
    
    Edmond não suporta a pressão causada pelas massivas perdas em sua vida e começa a agir de forma estranha, sendo constantemente atormentado pelas lembranças de seus entes queridos. Sendo considerado como louco, começa a receber uma série de medicações fortes que o deixa 'dopado' por meses a fio. A droga, que ainda estava em experimentação,  é tão forte que começa a afetar as lembranças de Edmond, que lentamente começa a se esquecer de todas as perdas bem como de quem ele próprio é, vivendo em um estágio próximo ao do vegetativo, apenas obedecendo ordens.
    
    Um dia, durante a manutenção do sistema de segurança do campo de concentração, uma pane ocorre no sistema elétrico e a segurança do complexo é comprometida. Praticamente todas as celas do complexo se destravam, o que causa uma tentativa de fuga em massa. Os guardas locais fazem o possível para conter os detentos, enquanto tentam comunicar-se com os campos próximos para solicitar reforços. Edmond, ainda meio sob o efeito dos medicamentos, tenta fugir também, sem atrair a atenção dos guardas. 
    
    Edmond precisará agora utilizar tudo que ainda se lembra para poder sobreviver. Isso significa se esconder nas sombras e atrás de objetos para evitar ser visto pelos guardas. Ou então, quando fugir do guarda não for opção, Edmond precisará eliminá-lo para prosseguir. O complexo em que ele se encontra possui 7 alas divididas em 3 andares, e para fugir ele precisará encontrar chaves específicas que dão acesso aos demais andares/alas até que ele alcance o salão central, que dá acesso aos jardins que por sua vez levam à uma densa floresta, possibilitando uma fuga mais “segura”.
    
    Além de lidar com os guardas, Edmond irá enfrentar outros seres, fantasmas que habitam sua mente. Esses fantasmas possuem características específicas que o relembram fatos importantes de sua vida. Esses fantasmas são, na verdade, suas lembranças de todas as mortes que o impactaram de alguma forma. No entanto, ele precisará lidar com essas lembranças e não só recordá-las como também aceitá-las para conseguir fugir do inferno em que se encontra. Caso contrário, ele será completamente consumido pela insanidade cada vez mais próxima dele, nesse ambiente completamente hostil e cheio de armadilhas, aguardando-o em cada sala. 

\section{Personagem Principal}
Edmond Gauthier, centro da trama do jogo, foi inspirado em soldados franceses que foram presos durante o início da Segunda Guerra Mundial e ficaram em campos de concentração para serem utilizados como moeda de troca. O personagem no início do jogo não consegue se lembrar direito de como ele foi parar naquele sanatório, por estar submetido à uma medicação forte o suficiente para afetar ainda mais a sua mente já deteriorada.

Durante todo o jogo o personagem mistura momentos de lucidez com momentos onde a loucura fala mais alto, que é quando os fantasmas surgem. Nas \textit{cutscenes} isso é mais evidente, onde a mente de Edmond oscila entre as recordações de seu passado, a realidade e a insanidade de sua mente, lutando contra seus fantasmas interiores à medida que tenta recuperar sua memória.

O personagem pode interagir com policiais que tentam conter os prisioneiros em fuga e com loucos que tentam fugir e matam quem encontrar pelo caminho. Ao encontrar um policial, o personagem poderá evitar entrar em seu campo de visão para não ser encontrado. Se o personagem escolher por assassinar o guarda, um fantasma do mesmo surgirá no local para atrapalhar o player, diminuindo sua vida e sanidade gradualmente à medida que permanece próximo à Edmond. Se encontrar um louco, ou o personagem deve tentar fugir o mais rápido possível ou abater o louco, já que o louco irá colocar sua vida em risco tão logo encontre Edmond.

\begin{figure}[h]
    \centering
    \caption{Concept art do Edmond}
     \includegraphics[keepaspectratio=true,scale=0.30]{images/Prev_Ed.png}
\end{figure}

\begin{figure}[h]
    \centering
    \caption{Imagem do Edmond \textit{in game}}
     \includegraphics[keepaspectratio=true,scale=4]{images/e_ig.png}
\end{figure}
\subsection{Características do personagem}
O personagem terá o tamanho de 2 \textit{tiles} posicionados verticalmente. Cada \textit{tile} possui 40x40px de tamanho. A tela ao todo terá 1280x720px. 

O personagem poderá correr, andar e andar agachado durante o jogo. Ao correr ele se locomoverá a 300px/s, andando ele se locomoverá à 130px/s e ao andar agachado ele se locomoverá a 75px/s.

O personagem poderá interagir com alguns itens durante o jogo, como armas, itens de recuperação de vida e blocos de papéis, todos descritos em seções mais à frente. Para interagir com esses itens, será utilizado os comandos já descritos na seção de controles.

\subsection{Habilidades do personagem}
O personagem possui as habilidades já descritas no tópico anterior: correr, agachar, andar agachado e rolar. Para correr, o personagem terá uma barra de resistência (barra de \textit{Stamina}) que irá determinar um tempo máximo para que o personagem permaneça correndo. Ao reduzir a barra de \textit{stamina} à zero, o personagem será incapaz de permanecer correndo por um tempo, tornando a andar em velocidade normal até que a barra carregue o suficiente para que ele volte a correr.

Todas as habilidades estarão disponíveis ao jogador desde o início do jogo, e o jogador poderá ver como utilizá-las ao acessar  o menu do jogo.

Como o objetivo do jogo será escapar do edifício, não será necessário o uso de transportes para locomoção durante o jogo. 

\subsection{\label{armas}Inventário}
O personagem terá acesso á diversas armas ao longo do jogo, no entanto todas serão de curto alcance. Para utilizá-las, o jogador deverá usar um dos dois tipos de ataque descritos anteriormente. A única arma que usará um controle diferenciado será a arma especial que é a seringa (o seu comando também já foi descrito na seção de controles). 

Os itens serão dispostos no jogo de forma aleatória, de forma que cada item terá uma determinada chance de aparecer de acordo com a fase em que o jogador se encontra. Para trocar de arma, basta que o jogador utilize a opção interagir com itens, descrita no tópico de controles. O jogador só poderá ter uma arma comum e uma arma especial por vez. O mesmo vale para o item de recuperação de vida e sanidade (tópicos que serão melhor detalhados mais a frente). As armas, bem como o dano causado por cada uma delas, será descrito abaixo.

As armas serão: o punho do personagem, faca, garrafa, cacetete e o item especial, seringa.

\subsection{Combate}
O personagem sempre atacará na área de um tile á sua frente. O movimento será sempre o mesmo, o que irá variar é a arma que ele carrega e o dano causado.

Ao ser atacado, o personagem não terá animação que represente o movimento de expressão de dor, mas o personagem irá piscar por alguns segundos e a barra de vida irá diminuir proporcionalmente ao dano, demonstrando que ele foi atacado.

Não existirão combos no jogo.

\section{\label{edmond} Os medidores de saúde do personagem}
\subsection{A vida}
\begin{itemize}
\item Aspectos gerais
\end{itemize}

A vida, aspecto fundamental do personagem, está posicionada no topo da tela, como descrito no HUD, é a primeira das três barras.

O jogo apresentará um item que irá recuperar parte da vida do personagem, o kit  médico. Esse item irá aparecer em localizações e quantidades aleatórias dentro do mapa. O kit médico irá recuperar em 20\% a vida do personagem.

\begin{itemize}
\item Vida e morte no jogo
\end{itemize}

A vida é afetada pelos fantasmas, pelos loucos e pelos policiais dentro do jogo. Como cada um irá afetar a vida do personagem será descrito em sessões mais a frente.

O personagem possui 3 vidas e elas são apresentadas no início de cada fase. Dentro do jogo, ela é representada pela barra vermelha que vai de 0 a 100\%. Ao ter a barra de vida reduzida a 0\% uma vida será descontada. Ao perder a vida, a partida é reiniciada e os itens conseguidos até o momento são mantidos

Ao perder as 3 vidas, a tela de \textit{Game Over} é apresentada e o jogo é reiniciado. Ao reiniciar a partida, o personagem irá perder todos os itens que tiver acumulado ao longo da partida.

\subsection{A \textit{Stamina}}
A barra de \textit{stamina} é a barra que define o quão apto o personagem está para correr. Ela estará posicionada abaixo da barra de vida, como mostrado anteriormente no tópico de Câmera e HUD.

A \textit{stamina}, ou 'resistência', do personagem irá reduzir gradativamente à medida que o personagem permaneça correndo. A barra irá diminuir em aproximadamente 14\% à cada segundo e, após zerada, ficará nesse estado por 3 segundos e então irá recomeçar a recuperar 6\% a cada segundo.

\subsection{A sanidade}
\begin{itemize}
\item O que é
\end{itemize}
A terceira e última barra apresentada no HUD tem importância igual ou até maior que a vida do personagem. Ela é o elemento principal do jogo pois interfere diretamente na mecânica do jogo. 

A barra de sanidade define quão \lq\lq mentalmente são\rq\rq\ o personagem se encontra. Ela pode ser reduzida ao entrar em contato com os fantasmas ou abater um guarda. 

\begin{itemize}
\item Efeitos \textit{in game}
\end{itemize}
Ao ter a sanidade reduzida abaixo dos 40\% a frequência da música mudará e a tela irá começar a tremer com uma frequência proporcional à redução da sanidade. Além disso, os fantasmas existentes no jogo irão se mexer com maior velocidade.

\begin{itemize}
\item Como recuperar a sanidade
\end{itemize}

A sanidade pode ser recuperada de duas formas: perdendo a vida durante o jogo ou recuperando através de um item presente no jogo: a pílula da sanidade.

A  pílula da sanidade recupera 30\% da sanidade do player. No entanto, por se tratar de um medicamento considerado \lq\lq forte\rq\rq o personagem terá uma redução na velocidade de 50\% por 10 segundos. Após esse período, o personagem voltará a ter a velocidade normal.

\pagebreak
\section{\label{characters} Principais Personagens do Mundo do Jogo}
O personagem principal do jogo é o Edmond Gauthier, no entanto, alguns outro personagens também compõe o universo do jogo, apresentados como inimigos do personagem. Essa seção apresenta esses personagens e um pouco de sua relação com o personagem.

Os fantasmas são, na verdade,personagens que só existem dentro da cabeça do Edmond. A função dos fantasmas é fazer com que Edmond perca completamente a sanidade e acabe se tornando mais uma vítima de suas próprias escolhas. No entanto, Edmond deve lutar para consiguir aceitar os fardos adquiridos ao longo da guerra e superar as perdas sofridas no processo.

\subsection{Fantasma dos soldados fuzilados}
Como previamente descrito na história do jogo, Edmond não consegue lidar com algumas mortes, como no caso de alguns soldados que, embora inimigos, não morreram de forma digna, segundo o pensamento do Edmond. 
	
	A forma do personagem tem o objetivo de lembrar o Edmond sobre o significado daquele personagem. Isso justifica os furos que o mesmo tem no corpo, representando os tiros disparados nos soldados fuzilados. 
	
	Esse fantasma aparece para o Edmond na fase 5 para impedir que Edmond continue avançando em sua fuga.

\textit{Nota: nesse jogo não haverão NPCs.}